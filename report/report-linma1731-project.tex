%%% ====================================================================
%%% @LaTeX-file{
%%%   filename  = "aomsample.tex",
%%%   copyright = "Copyright 1995, 1999 American Mathematical Society,
%%%                2005 Hebrew University Magnes Press,
%%%                all rights reserved.  Copying of this file is
%%%                authorized only if either:
%%%                (1) you make absolutely no changes to your copy,
%%%                including name; OR
%%%                (2) if you do make changes, you first rename it
%%%                to some other name.",
%%% }
%%% ====================================================================
\NeedsTeXFormat{LaTeX2e}% LaTeX 2.09 can't be used (nor non-LaTeX)
[1994/12/01]% LaTeX date must December 1994 or later
\documentclass[final]{aomart}
\usepackage[english]{babel}
%\usepackage[top=1.9cm, bottom=1.9cm, left=2.2cm, right=2.2cm]{geometry} % marges

\usepackage{mathtools,amssymb,amsthm}
\usepackage{bm}
\usepackage{float}
\usepackage{siunitx}
\usepackage{minted}
\usepackage{tikz}
\usepackage{pgfplots}

%    Some definitions useful in producing this sort of documentation:
\chardef\bslash=`\\ % p. 424, TeXbook
%    Normalized (nonbold, nonitalic) tt font, to avoid font
%    substitution warning messages if tt is used inside section
%    headings and other places where odd font combinations might
%    result.
\newcommand{\ntt}{\normalfont\ttfamily}
%    command name
\newcommand{\cn}[1]{{\protect\ntt\bslash#1}}
%    LaTeX package name
\newcommand{\pkg}[1]{{\protect\ntt#1}}
%    File name
\newcommand{\fn}[1]{{\protect\ntt#1}}
%    environment name
\newcommand{\env}[1]{{\protect\ntt#1}}
\hfuzz1pc % Don't bother to report overfull boxes if overage is < 1pc

%       Theorem environments

%% \theoremstyle{plain} %% This is the default
\newtheorem[{}\it]{thm}{Theorem}[section]
\newtheorem{cor}[thm]{Corollary}
\newtheorem{lem}[thm]{Lemma}
\newtheorem{prop}[thm]{Property}
\newtheorem{propo}[thm]{Proposition}
\newtheorem{ax}{Axiom}

\theoremstyle{definition}
\newtheorem{defn}{Definition}[section]
\newtheorem{rem}{Remark}[section]
\newtheorem*[{}\it]{notation}{Notation}
\newtheorem{step}{Step}

\numberwithin{equation}{section}

\newcommand{\thmref}[1]{Theorem~\ref{#1}}
\newcommand{\secref}[1]{\S\ref{#1}}
\newcommand{\lemref}[1]{Lemma~\ref{#1}}


%       Math definitions

\newcommand{\A}{\mathcal{A}}
\newcommand{\B}{\mathcal{B}}
\newcommand{\st}{\sigma}
\newcommand{\XcY}{{(X,Y)}}
\newcommand{\SX}{{S_X}}
\newcommand{\SY}{{S_Y}}
\newcommand{\SXY}{{S_{X,Y}}}
\newcommand{\SXgYy}{{S_{X|Y}(y)}}
\newcommand{\Cw}[1]{{\hat C_#1(X|Y)}}
\newcommand{\G}{{G(X|Y)}}
\newcommand{\PY}{{P_{\mathcal{Y}}}}
\newcommand{\X}{\mathcal{X}}
\newcommand{\wt}{\widetilde}
\newcommand{\wh}{\widehat}
\newcommand{\dig}{\psi^{(0)}}
\newcommand{\trig}{\psi^{(1)}}
\newcommand{\imagj}{\mathrm{j}\mkern1mu} % Imaginary unit but with 100% more engineering

\renewcommand{\epsilon}{\varepsilon}
\renewcommand{\theta}{\vartheta}
\renewcommand{\kappa}{\varkappa}
\renewcommand{\rho}{\varrho}
\renewcommand{\phi}{\varphi}

\newcommand{\like}{\mathcal{L}} % likelihood
\newcommand{\loglike}{\ell} % log-likelihood
\newcommand{\e}{\mathrm{e}} % euler's constant
\newcommand{\pdf}{f} % probability density function
\newcommand{\htheta}{\hat{\theta}} % estimator (abuse of notation)
\newcommand{\hTheta}{\wh{\Theta}} % estimator
\DeclareMathOperator{\newdiff}{d} % use \dif instead
\newcommand{\dif}{\newdiff\!} % differential operator
\newcommand{\fisher}{\mathcal{I}} % fisher information matrix
\DeclareMathOperator{\var}{var}
\DeclareMathOperator{\cov}{cov}
\DeclareMathOperator{\dom}{dom}
\DeclareMathOperator*{\argmax}{arg\,max}
\DeclareMathOperator*{\plim}{plim}

\makeatletter
\DeclareRobustCommand{\expe}{\mathbb{E}\@ifstar\@firstofone\@expe}
\newcommand{\@expe}[1]{\left[#1\right]}
\makeatother

\makeatletter
\DeclareRobustCommand{\var}{\mathbb{V}\@ifstar\@firstofone\@expe}
\newcommand{\@var}[1]{\left[#1\right]}
\makeatother

%    \interval is used to provide better spacing after a [ that
%    is used as a closing delimiter.
\newcommand{\interval}[1]{\mathinner{#1}}

%    Notation for an expression evaluated at a particular condition. The
%    optional argument can be used to override automatic sizing of the
%    right vert bar, e.g. \eval[\biggr]{...}_{...}
\newcommand{\eval}[2][\right]{\relax
  \ifx#1\right\relax \left.\fi#2#1\rvert}

%    Enclose the argument in vert-bar delimiters:
\newcommand{\envert}[1]{\left\lvert#1\right\rvert}
\let\abs=\envert

%    Enclose the argument in double-vert-bar delimiters:
\newcommand{\enVert}[1]{\left\lVert#1\right\rVert}
\let\norm=\enVert

%\setcounter{tocdepth}{5}

\title[Fish schools tracking]{LINMA1731 -- Project 2019\\
Fish schools tracking}

\author{Louis Navarre}
\address{Université catholique de Louvain, Ottignies-Louvain-la-Neuve, Belgium}
\fulladdress{École Polytechnique, Université catholique de Louvain, Place de l'Université 1, 1348 Ottignies-Louvain-la-Neuve, Belgique}
\email{navarre.louis@student.uclouvain.be}
\givenname{Louis}
\surname{Navarre}

\author{Gilles Peiffer}
\address{Université catholique de Louvain, Ottignies-Louvain-la-Neuve, Belgium}
\fulladdress{École Polytechnique\\
	Université catholique de Louvain\\
	Place de l'Université 1, 1348 Ottignies-Louvain-la-Neuve, Belgium}
\email{gilles.peiffer@student.uclouvain.be}
\givenname{Gilles}
\surname{Peiffer}

%\oldsubsections
\copyrightnote{\textcopyright~2019 Gilles Peiffer and Louis Navarre}

\begin{document}

\begin{abstract}
	In this paper we solve the first part of the project for the class ``Stochastic processes: Estimation and prediction'' given during the Fall term of 2019.
	The average speed of each fish in a school of fish is approximated by a gamma-distributed random variable with a shape parameter \(k\) and a scale parameter \(s\), and various methods for estimating this quantity are given; a numerical simulation is also included.
\end{abstract}

\maketitle
\tableofcontents

\part{Average speed estimation}
\section{Introduction}
For the purpose of this project, we assume that the speed of each fish in a school
at time \(i\) is a random variable \(V_i\) following a Gamma distribution, as suggested in [1].
This distribution is characterized by two parameters:
a shape parameter \(k > 0\) and a scale parameter \(s > 0\).
The parameters are the same for every fish and are time invariant.
The aim of this first part is to identify these two parameters using empirical observations \(v_i\).

\section{Maximum likelihood estimation}
Let \(v_i\) be i.i.d. realisations of a random variable following a Gamma distribution \(\Gamma(k, s)\) (with \(i = 1,\ldots, N)\).
We first assume that the shape parameter \(k\) is known.

We start by deriving the maximum likelihood estimator of \(\theta \coloneqq s\) based on \(N\) observations.
Since the estimand $\theta$ is a deterministic quantity, we use Fisher estimation.
In order to do this, let us restate the probability density function of \(V_i \sim \Gamma(k, s)\):
\begin{equation}
\pdf_{V_i}(v_i; k, s) = \frac{1}{\Gamma(k) s^k} v_i^{k-1} \e^{-\frac{v_i}{s}}\,, \quad i = 1, \ldots, N\,.\label{eq:gamma}
\end{equation}
With this in mind, we can find that the likelihood \(\like(v_1, \ldots, v_N; k, \theta)\) is given by
\begin{align}
\like(v_1, \ldots, v_N; k, \theta) &= \prod_{i=1}^{N} \pdf_{V_i}(v_i; k,\theta) = \prod_{i=1}^{N} \frac{1}{\Gamma(k) \theta^k} v_i^{k-1} \e^{-\frac{v_i}{\theta}}\,.
\end{align}
In order to alleviate notation, we compute instead the log-likelihood, which is generally easier to work with\footnote{This is possible because the values of \(\theta\) which maximize the log-likelihood also maximize the likelihood.}:
\begin{align}
\ell(v_1, \ldots, v_N; k, \theta) &\coloneqq \ln \like(v_1, \ldots, v_N; k, \theta)\\
&= \sum_{i=1}^{N} \ln\Bigg(\frac{1}{\Gamma(k) \theta^k} v_i^{k-1} \e^{-\frac{v_i}{\theta}}\Bigg)\\
&= (k-1) \sum_{i=1}^{N}\ln v_i - \sum_{i=1}^{N} \frac{v_i}{\theta} - N \big(k \ln \theta + \ln \Gamma(k)\big)\,.\label{eq:loglikelihood}
\end{align}
Now, in order to obtain the maximum likelihood estimate \(\htheta\), we must differentiate the log-likelihood with respect to the estimand \(\theta\), and set it equal to zero:
\begin{align}
\eval{\frac{\partial \ell(v_1, \ldots, v_N; k, \theta)}{\partial \theta}}_{\theta = \htheta} &= -\frac{kN}{\htheta} + \frac{\sum_{i=1}^{N} v_i}{\htheta^2} = 0\\
\iff \htheta &= \frac{\sum_{i=1}^{N} v_i}{kN} = \frac{\widebar{v}}{k}\,.\\
\intertext{This then allows us to find the maximum likelihood estimator \(\hTheta\), given by}
\hTheta &= \frac{\sum_{i=1}^{N} V_i}{kN} = \frac{\widebar{V}}{k}\,.\label{eq:mlestimator}
\end{align}

\section{Properties of the estimator}

We now wish to show some of the properties of this estimator.
The definitions of these properties are given in Appendix~\ref{app:defs}.
\begin{prop}
	The maximum likelihood estimator derived in \eqref{eq:mlestimator} is asymptotically unbiased, that is,
	\begin{equation}
	\lim_{N \to +\infty} \expe{g(V_1, \ldots, V_N); \theta} = \theta\,.
	\end{equation}
\end{prop}
\begin{proof}
	We wish to prove that \(\lim_{N \to +\infty} \expe{\frac{\widebar{V}}{k}} = \theta\).
	We recall that \(\expe{V_i} = k \theta\) for \(V_i \sim \Gamma(k, \theta)\)
	and that the expected value operator is linear to obtain
	\begin{equation}
	\expe{\frac{\widebar{V}}{k}} = \frac{\expe{\frac{1}{N} \sum_{i=1}^N V_i}}{k} = \frac{\frac{1}{N} \sum_{i=1}^N \expe{V_i}}{k} = \frac{\frac{1}{N} N k \theta}{k} = \theta\,.
	\end{equation}
	This proves that the maximum likelihood estimator of \eqref{eq:mlestimator} is unbiased,
	hence it is also asymptotically unbiased.
\end{proof}

\begin{prop}
	\label{prop:eff}
	The maximum likelihood estimator derived in \eqref{eq:mlestimator} is efficient.
\end{prop}
\begin{proof}
	We use the fact that the random variables are independent to simplify the computations.
	Since \(\theta\) is a scalar parameter, the Fisher information matrix is a scalar, equal to
	\begin{align}
	\fisher(\theta) &= - N \expe{\frac{\partial^2}{\partial \theta^2} \Bigg((k-1) \ln v_1 - \frac{v_1}{\theta} - \big(k \ln \theta + \ln \Gamma(k)\big)\Bigg)}\\
	&=  N\expe{\frac{\partial^2}{\partial \theta^2} \Big(\frac{v_1}{\theta} + k \ln \theta \Big)} = \frac{kN}{\theta^2}\,.\label{eq:crlb}
	\end{align}
	We must also compute the variance of the ML estimator \(\hTheta\), which is given by
	\begin{equation}
	\var{\hTheta} = \var{\frac{\overline{V}}{k}} = \frac{\theta^2}{kN}\,.
	\end{equation}
	The Cramér--Rao lower bound is thus reached for all values of \(\theta\), which concludes the proof.
\end{proof}

\begin{prop}
	The maximum likelihood estimator of \eqref{eq:mlestimator} is best asymptotically normal.
\end{prop}
\begin{proof}
	In our case, we can show using the Cramér--Rao lower bound that \(\Sigma\) is minimal if it is equal to \(\fisher^{-1}(\theta)\).
	To alleviate notations, we will write \(\ell(\theta)\) instead of \(\ell(v_1, \ldots, v_N; k, \theta)\).
	By definition, since \(\htheta = \argmax_{\theta} \ell(\theta)\),
	we know that \(\ell'(\htheta) = 0\).
	Let \(\theta_0\) be the true value of the parameter \(\theta\).
	We can then use Taylor expansion on \(\ell'(\htheta)\) around \(\htheta = \theta_0\) to obtain
	\begin{align}
	\ell'(\htheta) &= \ell'(\theta_0) + \frac{\ell''(\theta_0)}{1!} (\htheta - \theta_0) + \mathcal{O}\left((\htheta-\theta_0)^2\right)\,.
	\intertext{We know the expression on the left is zero, hence}
	\ell'(\theta_0) &= -\ell''(\theta_0) (\htheta - \theta_0) + \mathcal{O}\left((\htheta-\theta_0)^2\right)\,.
	\intertext{Rearranging and multiplying by \(\sqrt{n}\), we get}
	\sqrt{n}(\htheta - \theta_0) &= \frac{\ell'(\theta_0)/\sqrt{n}}{-\ell''(\theta_0)/n + \mathcal{O}\left((\htheta-\theta_0)/n\right)}\,.
	\end{align}
	
	Next, we need to show that \(\frac{1}{\sqrt{n}} \ell'(\theta_0) \sim \mathcal{N}\big(0, \fisher(\theta_0)\big)\).
	This is done using the Lindeberg--Lévy central limit theorem, in Appendix~\ref{app:banproof}.
	We know that \(\frac{1}{N} \ell''(\theta_0) = \fisher(\theta_0)\).
	Finally, we can rewrite
	\begin{equation}
	\sqrt{N} (\htheta - \theta_0) \sim \frac{\mathcal{N}(0, \fisher(\theta_0))}{\fisher(\theta_0)} = \mathcal{N}(0, \fisher^{-1}(\theta_0))\,,
	\end{equation}
	where we didn't take into account the remainder of the Taylor series, which goes to zero.
	This proves that the ML estimator is best asymptotically normal.
\end{proof}

\begin{prop}
	The maximum likelihood estimator of \eqref{eq:mlestimator} is consistent.
\end{prop}
\begin{proof}
	We have shown that the estimator is unbiased, hence its MSE is equal to its variance.
	Since the estimator is efficient by Property~\ref{prop:eff}, we know that its variance is equal to the Cramér--Rao lower bound, \(\cov \hTheta = \fisher^{-1}(\theta)\).
	We found this lower bound to be equal to \(\frac{\theta^2}{kN}\) in \eqref{eq:crlb}.
	We have
	\begin{equation}
	\lim_{N \to +\infty} \cov \hTheta = \lim_{N \to +\infty} \frac{\theta^2}{kN} = 0\,.
	\end{equation}
	This proves that the variance (and hence the mean square error) of the estimator goes to zero as \(N\) goes to infinity, hence the estimator is consistent.
\end{proof}

\section{Joint maximum likelihood estimation}
\label{sec:joint}
We now consider \(V_i \sim \Gamma(k, s)\) (for \(i = 1,\ldots,N)\) with both \(k\) and \(s\) unknown.
Before, we assumed \(k\) was known, so we could maximize the log-likelihood function with respect to \(s\).
Now, we have to maximize this function with respect to \(s\) and \(k\) at the same time.
We know the maximum likelihood estimator of \(s\), \(\hat{s} = f(k)\).
Therefore, in the log-likelihood function, we can replace all the occurrences of \(s\) by the estimator we found, \(\hat{s}\).
One then gets a function of \(k\) only, which can be differentiated and its derivative set to zero.
Solving this, one can find the maximum likelihood estimator of \(k\).
The log-likelihood is used as given in \eqref{eq:loglikelihood}.
We abusively write \(\ell(v)\) instead of \(\ell(v_1, \ldots, v_N; \theta)\).
Substituting in the estimator \(\hat{s}\) instead of \(s\), one finds
\begin{align}
\ell(v) = (k-1) \sum_{i=1}^{N}\ln v_i - \sum_{i=1}^{N} \frac{kv_i}{\widebar{v}} - N k \ln \widebar{v} + N k \ln k - N \ln \Gamma(k)\big)\,.
\end{align}
Taking the derivative of this function with respect to \(k\), we get
\begin{align}
\eval{\frac{\partial \ell(v)}{\partial k}}_{k = \hat{k}} &= \sum_{i=1}^N \ln v_i - N - N \ln \widebar{v} + N \ln \hat{k} + \frac{N\hat{k}}{\hat{k}} - N\frac{\Gamma(\hat{k})}{\Gamma(\hat{k})}\dig(\hat{k})\\
&= \sum_{i=1}^N \ln v_i - N \ln \sum_{i=1}^N v_i + N \ln \hat{k}  + N \ln N - N \dig(\hat{k})\,,
\end{align}
where \(\dig\) is the digamma function, i.e. the logarithmic derivative of the gamma function.
One must now look for a root of this equation:
\begin{align}
\ln \hat{k} - \dig(\hat{k}) &= \ln \left(\sum_{i=1}^N v_i\right) - \ln N - \frac{\sum_{i=1}^N \ln v_i}{N}\\
\iff \ln \hat{k}  - \dig(\hat{k}) &= \ln \left(\frac{\sum_{i=1}^N v_i}{N}\right) - \frac{\sum_{i=1}^N \ln v_i}{N}\,.\label{eq:mlk}
\end{align}
This equation has no closed-form solution for \(\hat{k}\), but can be approximated using numerical methods since the function is very well-behaved.

\section{Numerical simulation}
For the numerical simulation, \(N\) random variables were generated from a distribution with parameters \(\Gamma(1, 2)\), for different values of \(N\) (\mintinline{julia}{10:15:1000}).
For each value of \(N\), the experiment was repeated \(M = 500\) times.

In order to use method of moments estimation for the Gamma distribution given in \eqref{eq:gamma},
one first needs to know its characteristic function: \(\phi_{V_i}(t) = \expe{\e^{\imagj t V_i}} = (1 - \imagj s t)^{-k}\).
The \(n\)th moment is given by \(\mu_n = \expe{V_i^n} = \imagj^{-n} \phi_{V_i}^{(n)}(0)\).
Since the parameter vector has dimension two,
we need to compute the first two moments.
These are given by \(\mu_1 = ks\) and \(\mu_2 = k(k+1) s^2\).
One can use the sample moments \(\hat{\mu}_1\) and \(\hat{\mu}_2\)
to estimate \(\mu_1\) and \(\mu_2\) as follows:
\begin{equation}
\hat{\mu}_1 = \frac{1}{N} \sum_{i=1}^N v_i\,,\quad \hat{\mu}_2 = \frac{1}{N} \sum_{i=1}^N v_i^2\,.
\end{equation}
Using some simple algebra, one then finds
\begin{equation}
\hat{k}_{\textnormal{MOM}} = \frac{\hat{\mu}_1^2}{\hat{\mu}_2 - \hat{\mu}_1^2}\,, \quad \hat{s}_{\textnormal{MOM}} = \frac{\hat{\mu}_2}{\hat{\mu}_1} - \hat{\mu}_1\,.
\end{equation}

The maximum likelihood estimators are computed using the formulas in \eqref{eq:mlestimator} and \eqref{eq:mlk}, for the given sample.
As mentioned in Section~\ref{sec:joint},
the maximum likelihood estimator for \(k\) has no closed-form solution,
but can be approximated using numerical methods which require an initial guess.
One such first guess is provided by the method of moments estimator for the parameter,
that is
\begin{equation}
\hat{k}_{\textnormal{ML}}^{(0)} = \hat{k}_{\textnormal{MOM}} =  \frac{\hat{\mu}_1^2}{\hat{\mu}_2 - \hat{\mu}_1^2}\,.
\end{equation}
Another possible choice for the first guess is
\begin{equation}
\hat{k}_{\textnormal{ML}}^{(0)} = \frac{3 - \xi + \sqrt{(\xi - 3)^2 + 24\xi}}{12 \xi}\,, \quad \textnormal{where } \xi = \ln \widebar{v} + \frac{1}{N} \sum_{i=1}^N \ln v_i\,.
\end{equation}
This guess can be shown to be within \(\SI{1.5}{\percent}\) of the actual value.
The estimator for \(s\) can then be found from \eqref{eq:mlestimator}, using \(\hat{k}_{\textnormal{ML}}\) instead of \(k\).

\begin{figure}[H]
	\centering
	\includegraphics[width=\textwidth]{img/k.png}
	\caption{Estimators for the first parameter.}
	\label{fig:k_est}
\end{figure}
\begin{figure}[H]
	\centering
	\includegraphics[width=\textwidth]{img/s.png}
	\caption{Estimators for the second parameter.}
	\label{fig:s_est}
\end{figure}

On Figures~\ref{fig:k_est} and \ref{fig:s_est}, the mean and standard deviation are shown for the \(M\) values of both the method of moments estimator and the maximum likelihood estimator, for different values of the size of the sample, \(N\).
For both parameters, both estimators are unbiased, but the ML estimator has a lower variance.
This should not come as a surprise: since it is efficient, every other estimator must have a greater or equal asymptotic variance.
As is shown numerically in Section~\ref{sec:ratio},
this variance asymptotically goes to \(\fisher^{-1}(\theta)\),
the Cramér--Rao lower bound.
This is also expected, in light of Property~\ref{prop:eff}.

\section{Fisher information matrix}
We can compute the Fisher information matrix. Since we have two estimators, this matrix is a $2\times 2$ matrix. This matrix is hence computed by \label{fisher_matrix}
\[ 
\fisher(\theta) = \left( \begin{array}{cc}
-\expe{\frac{\partial^2 l(x)}{\partial s^2}} & -\expe{\frac{\partial^2 l(x)}{\partial k \partial s}} \\
-\expe{\frac{\partial^2 l(x)}{\partial s \partial k}} & -\expe{\frac{\partial^2 l(x)}{\partial k^2}}  \\
\end{array} \right)\] 
And remember the log-likelihood function, computed before.
\begin{align}
	\ell(x) = (k-1) \sum_{i=1}^{N}\ln v_i - \sum_{i=1}^{N} \frac{v_i}{s} - N \big(k \ln s + \ln \Gamma(k)\big)\,
\end{align}
The calculation of theses entries can be found in appendix \ref{fisher_information}. Finally, the Fisher information matrix is 
\[ \fisher(\theta) = N\left( \begin{array}{cc}
\frac{K}{s^2} & \frac{1}{s} \\
\frac{1}{s} & \psi_1(k) \\
\end{array} \right)\] 
And the inverse of this matrix, the Cramr-Rao lower-bound, is
\[ \fisher^{-1}(\theta) = \frac{1}{N}\left( \begin{array}{cc}
\psi_1(k) & -\frac{1}{s}\\
-\frac{1}{s} & \frac{K}{s^2}
\end{array} \right)\cdot \frac{s^2}{K\psi_1(k) - 1} \] 

\section{Numerical proof}
\label{sec:ratio}
Figure \ref{fig:CRLB} show the matrix norm induced by the vector 2-norm of the \emph{ratio} matrix. This matrix is is the result of the ratio of the entries of the empirical covariance matrix by the inverse of the Fisher information matrix, subtracted by one. As a result, if the empirical covariance matrix tends to reach the Cramr-Rao lower bound, the ratio of the entries should tend to 1, and hence the matrix induced norm of this matrix subtracted by 1 should be equal to 0. Indeed, we see on figure \ref{fig:CRLB} that this matrix induced norm tends to 0 as the size of the sample vector increases. It empirically proofs that our estimators reach the Cramr-Rao lower bound, meaning that they are efficient.

\begin{figure}[H]
	\centering
	\begin{tikzpicture}
\begin{axis}[grid={major}, title={\Large Convergence towards the Cramér--Rao bound}, xlabel={\footnotesize Size of the sample vector}, ylabel={\footnotesize Spectral norm of $\cov \hTheta \oslash \mathcal{I}^{-1}(\vartheta) - \left(\begin{smallmatrix}1&1\\1&1\end{smallmatrix}\right)$}, xmode={log}]
    \addplot
        table[row sep={\\}]
        {
            x  y  \\
            10.0  2.6715124732040203  \\
            50.0  0.25407679273353834  \\
            150.0  0.07881648895451268  \\
            3000.0  0.009691607578209131  \\
        }
        ;
\end{axis}
\end{tikzpicture}

	\caption{Using an adequately-centered spectral norm, one can visualize to which extent the Cramér--Rao lower bound is reached.
	The symbol ``\(\oslash\)'' denotes Hadamard (element-wise) division.}
	\label{fig:CRLB}
\end{figure}

\appendix

\section{Definitions of properties}
\label{app:defs}

\begin{defn}[Unbiased estimator]
	\label{def_unbiased}
	The Fisher estimator \(\hTheta = g(Z)\) of \(\theta\) is \emph{unbiased} if
	\begin{equation}
	m_{\hTheta; \theta} \coloneqq \expe{g(Z); \theta} = \theta\,, \quad \textnormal{for all } \theta\,.
	\end{equation}
\end{defn}

\begin{thm}[Cramér--Rao inequality]
	If \(Z = (Z_1, \ldots, Z_N)^T\) with i.i.d. random variables \(Z_k\) and if its probability density function given by \(\pdf_Z(z; \theta) = \prod_{k=1}^{N} \pdf_{Z_k}(z_k; \theta)\) satisfies certain regularity conditions, then the covariance of any unbiased estimator \(\hTheta\) satisfies the \emph{Cramér--Rao inequality}
	\begin{equation}
	\cov \hTheta \succeq \fisher^{-1}(\theta)\,,
	\end{equation}
	where \(\fisher(\theta)\) is the \(p \times p\) \emph{Fisher information matrix},
	defined by
	\begin{equation}
	\big[\fisher(\theta)\big]_{i, j} \coloneqq -\expe{\frac{\partial^2 \ln \pdf_Z(z; \theta)}{\partial \theta_i \partial \theta_j}}\,.
	\label{information_matrix}
	\end{equation}
\end{thm}
\begin{defn}[Efficient estimator]
	\label{def_eff}
	An unbiased estimator is said to be \emph{efficient} if it reaches the Cramér--Rao bound for all values of \(\theta\), that is,
	\begin{equation}
	\cov \hTheta = \fisher^{-1}(\theta)\,, \quad \forall \theta\,.
	\end{equation}
\end{defn}

\begin{defn}[Best asymptotically normal estimator]
	\label{def_normal}
	A sequence \(\{\hTheta_N(Z)\}_{N \in \mathbb{N}}\) of consistent estimators of \(\theta\) is called \emph{best asymptotically normal} if
	\begin{equation}
	\sqrt{N} \left(\hTheta_N(Z) - \theta\right) \xrightarrow[N \to +\infty]{\mathcal{D}} \mathcal{N}(0, \Sigma)\,,
	\end{equation}
	for some minimal positive definite matrix \(\Sigma\).
\end{defn}

\begin{defn}[Consistent estimator]
	\label{def_consis}
	A sequence \(\{\hTheta_N(Z)\}_{N \in \mathbb{N}}\) of estimators of \(\theta\) is called \emph{consistent} if
	\begin{equation}
	\plim_{N \to +\infty} \hTheta_N(Z) = \theta\,.
	\end{equation}
	Equivalently, one can show that the MSE of the estimator converges to zero as \(N\) goes to infinity.
\end{defn}

\section{Omitted proofs}
The following is a proof of normality, using the Lindeberg--Lévy formulation of the central limit theorem.
\begin{proof}
\label{app:banproof}
We want to prove that  \(\ell'(\theta_0)/\sqrt{N} \sim \mathcal{N}\big(0, \fisher(\theta_0)\big)\).
We simplify notation by writing \(\pdf(v_i)\) instead of \(\pdf_{V_i}(v_i; k, \theta)\)
and using Euler's notation for derivatives.
First, we show that the expected value of \(\ell'(\theta_0)/\sqrt{N}\) is zero.
\begin{align}
\expe{\frac{\ell'(\theta_0)}{\sqrt{N}}} &= \int_{-\infty}^{+\infty} \partial_{\theta_0} \left(\sum_{i=1}^N \frac{\ln \pdf(v_i)}{\sqrt{N}}\right) \pdf(v_i)\dif v_i\\
&= \frac{1}{\sqrt{N}} \sum_{i=1}^N \int_{-\infty}^{+\infty} \frac{\partial_\theta \pdf(v_i)}{\pdf(v_i)} \pdf(v_i)\dif v_i\\
&= \frac{1}{\sqrt{N}} \sum_{i=1}^N \int_{-\infty}^{+\infty} \frac{\partial_\theta \pdf(v_i)}{\pdf(v_i)} \pdf(v_i) \dif v_i\\
&= \frac{1}{\sqrt{N}} \sum_{i=1}^N \int_{-\infty}^{+\infty} \partial_\theta \pdf(v_i) \dif v_i\\
&= \frac{1}{\sqrt{N}} \sum_{i=1}^N \partial_\theta \int_{-\infty}^{+\infty} \pdf(v_i) \dif v_i\\
&= 0\,.
\end{align}

Next, we compute the variance of \(\ell'(\theta_0)/\sqrt{N}\).
First, we compute (for \(i = 1, \ldots, N\))
\begin{align}
\expe{\big(\partial_{\theta_0} \ln \pdf(v_i)\big)^2} &= \int_{-\infty}^{+\infty} \partial_{\theta_0} \ln \pdf(v_i) \frac{\partial_{\theta_0} \pdf(v_i)}{\pdf(v_i)} \pdf(v_i)\dif v_i\\
&= \int_{-\infty}^{+\infty} \partial_{\theta_0} \ln \pdf(v_i) \partial_{\theta_0} \pdf(v_i) \dif v_i\,.
\intertext{Using the product rule, we can then find}
&= \begin{multlined}[t]
- \int_{-\infty}^{+\infty} \partial_{\theta_0 \theta_0} \ln \pdf(v_i) \pdf(v_i) \dif v_i \\
+ \int_{-\infty}^{+\infty} \partial_{\theta_0} \big(\partial_{\theta_0} \ln \pdf(v_i) \pdf(v_i)\big) \dif v_i\,.
\end{multlined}\\
&= - \expe{\partial_{\theta_0 \theta_0 } \ln \pdf(v_i)} + \partial_{\theta_0} \int_{-\infty}^{\infty} \frac{\partial_{\theta_0} \pdf(v_i)}{\pdf(v_i)} \pdf(v_i) \dif v_i\\
&= \fisher(\theta)\,,
\end{align}
where the last expression can be shown to be zero by a similar argument as the one used above for the expected value.
Knowing this, one easily finds
\begin{equation}
\var{\frac{\ell'(\theta_0)}{\sqrt{N}}} = \frac{1}{N} \var{\sum_{i=1}^N \partial_{\theta_0} \ln \pdf(v_i)} = \fisher(\theta_0)\,,
\end{equation}
since the random variables are i.i.d..
Using the Lindeberg--Lévy CLT, we thus have
\begin{equation}
\frac{\ell'(\theta_0)}{\sqrt{N}} \sim \mathcal{N}(0, \fisher(\theta_0))\,.\qedhere
\end{equation}

\end{proof}

\section{Calculations of the Fisher information matrix}
\label{fisher_information}
\begin{align}
\frac{\partial h(z;\theta)}{\partial s} & = \frac{1}{s^2}\sum_{i=1}^{N}v_i - \frac{Nk}{s}
\end{align}
\begin{align}
\frac{\partial h(z;\theta)}{\partial k} & = \sum_{i=1}^{N}\ln v_i - N\ln s - N\psi_0(k)
\end{align}
\begin{align}
\frac{\partial^2h(z;\theta)}{\partial s^2} & = -\frac{2}{s^3}\sum_{i=1}^{N}v_i + \frac{Nk}{s^2}
\label{00}
\end{align}
\begin{align}
\frac{\partial^2 h(z; \theta)}{\partial k^2} & = -N\psi_1(k).
\label{11}
\end{align}
\begin{align}
\frac{\partial^2 h(z; \theta)}{\partial k\partial s} & = -\frac{N}{s} .
\end{align}
Then, we take the expectation of these computed values.
\begin{align}
\fisher_{00} & = -\mathbb{E}\bigg\{ \frac{\partial^2 h(z;\theta)}{\partial s^2} \bigg\} = -\mathbb{E}\bigg\{ -\frac{2}{s^3}\sum_{i=1}^{N}v_i + \frac{Nk}{s^2} \bigg\}\\
& = \frac{2}{s^3}Nsk - \frac{NK}{s^2} = \frac{NK}{s^2}.
\end{align}
\begin{align}
\fisher_{01} & = \fisher_{10} =  -\mathbb{E}\bigg\{ \frac{\partial^2 h(z;\theta)}{\partial k \partial s} \bigg\} = \frac{N}{s}.
\end{align}
\begin{align}
\fisher_{11} & = -\mathbb{E}\bigg\{ \frac{\partial^2 h(z;\theta)}{\partial k^2} \bigg\} = -N\psi_1(k).
\end{align}
\bibliography{report-linma1731-project}
\bibliographystyle{aomalpha}

\end{document}
\endinput
